\documentclass[a4paper,11pt]{article}

\usepackage[MeX]{polski}
\usepackage[polish]{babel}
\usepackage[T1]{fontenc}
\usepackage[utf8]{inputenc}
\usepackage{hyperref}

\hypersetup{
	unicode=true,
	pdfborder=0 0 0,
	pdfstartview=FitH,
	pdfauthor={Bartosz Pieńkowski},
	pdftitle={Logowanie kluczem USB}
}

\author{Bartosz Pieńkowski}
\title{Logowanie kluczem USB}

\begin{document}
\maketitle

\section{Założenia projektu}
Celem projektu jest umożliwienie uwierzytelniania użytkowników systemu na podstawie kluczy USB posiadających unikalny numer seryjny. 

Obecność urządzenia o~numerze seryjnym przyporządkowanym do danego użytkownika eliminuje konieczność podawania przez niego hasła w~procesie logowania.

Opcjonalnie weryfikacja może dotyczyć zarówno obecności klucza USB o~żądanym numerze seryjnym, jak i~poprawności hasła, co pozwala zwiększyć bezpieczeństwo systemu.


\section{Mechanizm PAM}
\subsection{Zasada działania}
Optymalnym rozwiązaniem kwestii uwierzytelniania jest wykorzystanie mechanizmu PAM\footnote{Pluggable Authentication Modules} będącego zbiorem modułów oferujących różne metody uwierzytelniania, które można wykorzystać w~wymagających tego aplikacjach.

Modularna architektura mechanizmu gwarantuje elastyczność w~doborze metod uwierzytelniania dla każdej aplikacji z~osobna, dokonywanym poprzez modyfikację plików konfiguracyjnych zlokalizowanych w~katalogu \verb|/etc/pam.d/|.


\subsection{Moduł \texttt{pam\_usbkey.so}}
Realizacją założeń projektu jest dostarczenie modułu PAM implementującego uwierzytelnianie na podstawie numerów seryjnych podłączonych urządzeń USB. Założenia te spełnia moduł \verb|pam_usbkey.so|.

W momencie zainicjowania procesu uwierzytelniania moduł ten odczytuje numer seryjny przypisany uwierzytelnianemu użytkownikowi z~pliku konfiguracyjnego \verb|usbkey.conf|.

Kolejnym krokiem jest uzyskanie listy obecnych w~systemie urządzeń USB i~porównanie przypisanych im numerów seryjnych z~wartością odczytaną z~pliku konfiguracyjnego. Uwierzytelnienie kończy się sukcesem w~przypadku natrafienia na szukany numer seryjny, w~przeciwnym razie rezultatem jest odmowa dostępu.


\subsection{Konfiguracja}
Każda aplikacja korzystająca z~mechanizmu PAM posiada oddzielny plik konfiguracyjny w~katalogu \verb|/etc/pam.d/|, definiujący dobór modułów oraz wpływ wyniku ich działania na końcowy rezultat procesu uwierzytelniania.

Wymuszenie użycia modułu \verb|pam_usbkey.so| jako wystarczającego do pomyślnego zakończenia procesu uwierzytelniania wymaga modyfikacji pliku \verb|login| poprzez dodanie na szczycie stosu modułów linii:
\begin{quote}
	\verb|auth      sufficient      pam_usbkey.so|
\end{quote}

Inną możliwością jest wykorzystanie modułu \verb|pam_usbkey.so| razem ze standardową metodą uwierzytelniania hasłem, zaimplementowaną w~postaci modułu \verb|pam_unix.so|. W~tym wypadku proces uwierzytelniania kończy się sukcesem jedynie w~sytuacji, gdy obydwa moduły zakończą działanie z~pozytywnym rezultatem.

Narzuca to konieczność uzupełnienia pliku \verb|login| o~następujące linie:
\begin{quote}
	\begin{verbatim}
	auth      required        pam_usbkey.so
	auth      required        pam_unix.so
	\end{verbatim}
\end{quote}


\section{Instalacja modułu}
Kompilacja kodu źródłowego oraz instalacja modułu w~systemie odbywa się przy użyciu skryptu \emph{Makefile}, poprzez wydanie kolejno poleceń:
\begin{verbatim}
make
make install
\end{verbatim}

Za przywrócenie zawartości katalogu do stanu sprzed kompilacji odpowiada polecenie:
\begin{verbatim}
make clean
\end{verbatim}


\section{Plik konfiguracyjny}
Plik konfiguracyjny modułu \verb|pam_usbkey.so| przechowywany jest w~katalogu \linebreak \verb|/etc/security/| pod nazwą \verb|usbkey.conf|. Stanowi on przyporządkowanie numerów seryjnych urządzeń USB użytkownikom systemu.

Każda linia pliku składa się z~pary \verb|nazwa_użytkownika:numer_seryjny|. Przykładowy plik \verb|usbkey.conf| wygląda zatem następująco:
\begin{quote}
	\begin{verbatim}
	pienkowb:F0CADC673DF2C8ED
	kowalskj:066B0969C9ACEBE9
	wysockip:2DE087A50A1B42D7
	\end{verbatim}
\end{quote}

Istotną kwestią bezpieczeństwa jest nadanie plikowi ograniczonych praw dostępu w~celu zapobieżenia jego modyfikacji przez osoby nieuprawnione. Aktualizacja pliku leży w~gestii administratora systemu.


\begin{thebibliography}{}
	\bibitem{howto} \href{http://www.faqs.org/docs/Linux-HOWTO/User-Authentication-HOWTO.html\#AEN101}
		{Hernberg P.: \emph{User Authentication HOWTO}}
	\bibitem{guide} \href{http://www.centos.org/docs/5/html/Deployment\_Guide-en-US/ch-pam.html}
		{\emph{Red Hat Enterprise Linux Deployment Guide}}
	\bibitem{article} \href{http://www.packtpub.com/article/development-with-pluggable-authentication-modules-pam}
		{Geisshirt K.: \emph{Development with Pluggable Authentication Modules}}
\end{thebibliography}

\end{document}